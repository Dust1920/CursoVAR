\documentclass[oneside,spanish]{amsbook}
\usepackage[T1]{fontenc}
\usepackage[latin9]{inputenc}
\usepackage{geometry}
\geometry{verbose,tmargin=2cm,bmargin=2cm,lmargin=2cm,rmargin=2cm}
\setcounter{secnumdepth}{2}
\setcounter{tocdepth}{2}
\usepackage{amstext}
\usepackage{amsthm}

\makeatletter
%%%%%%%%%%%%%%%%%%%%%%%%%%%%%% Textclass specific LaTeX commands.
\numberwithin{section}{chapter}
\numberwithin{equation}{section}
\numberwithin{figure}{section}

\makeatother

\usepackage{babel}
\addto\shorthandsspanish{\spanishdeactivate{~<>.}}
\deactivatequoting

\begin{document}
\title{Modelación de Problemas y paso a Código.}

\maketitle


\chapter{Introducción}

La programación se ha vuelto una herramienta fundamental al momento
de abordar problemas, en nuestro caso, de optimización pesados. Esto
por la dificultad de encontrar soluciónes análiticas. Sin embargo,
gran parte de la resolución del problema, una vez se garantiza la
solución de su existencia, es la construcción del método numérico. 

\section{Notación. }

En el momento de la resolución de los problemas vamos a definir algunos
elementos. Notemos que cada problema de control óptimo $P$ puede
verse como una tetrada
\[
P=\left(f,\mathcal{X},\mathcal{A},C\right),
\]

donde $f$ es la dinámica del problema, $\mathcal{X}$ el espacio
de estados, $\mathcal{A}$ es el conjunto de acciones asociadas a
$P$ y $C$ es la función de costo o recompensa según lo requiera
$P.$ Con lo anterior podemos definir $A\left(x\right)$ como el conjunto
de acciones admisibles 
\[
A\left(x\right)\subset\mathcal{A},x\in\mathcal{X}.
\]
Para los problemas que abarcaremos nuestrá dinámica $f$ va a ser
de tal forma que 
\[
x_{k+1}=f\left(x_{k},a_{k}\right).
\]

Además, definiremos una situación como la colección 
\[
\mathcal{S}\left(x_{0},\pi\right)=\left\{ \left(x_{k},a_{k}\right)\right\} _{k=0}^{\infty},
\]

donde $\pi=\left(a_{0},\ldots\right)$ y $x_{k+1}=f\left(x_{k},a_{k}\right)$.
Importante mencionar que $\pi$ puede ser finita, lo que haría a $\mathcal{S}$
finita. Motivado por lo anterior denotaremos el costo total por 
\[
C\left(x_{0},\pi\right)=\sum_{k=0}^{\infty}c(x_{k},a_{k})
\]

recordando que si $\pi$ es finita. 
\[
C\left(x_{0},\pi\right)=c_{N}(x_{N})+\sum_{k=0}^{N-1}c(x_{k},a_{k}).
\]


\section{Metodos a implementar. }

Dentro de los problemas considerados, trabajaremos dos metodologías
similares. El algoritmo de programación dinámica, que de forma resumida,
consiste en defnir funciones $J_{k}$ tales que 
\begin{align*}
J_{N}\left(x\right) & =c_{N}\left(x\right)\\
J_{k+1}\left(x\right) & =\min_{a\in A\left(x\right)}\left\{ c\left(x,a\right)+J_{k}\left(f(x,a)\right)\right\} ,
\end{align*}
entonces la política óptima esta dada por una colección de funciones
$h_{k}$, donde 

\[
h_{k}\left(x\right)=\arg\min_{a\in A(x)}\left\{ c\left(x,a\right)+J_{k}\left(f(x,a)\right)\right\} .
\]

Sin embargo, está solo funciona cuando las politicas son finitas.
Para el caso infinito (horizonte infinito), en particular el caso
descontado, es decir, 
\[
C=\sum_{k=0}^{\infty}\beta^{k}c(x_{k},a_{k}),0<\beta<1
\]
 tenemos el algoritmo dado por las ecuación de Bellman, que en forma
resumida, trata de construir unas funciones $w_{k}$ donde $w_{0}$
es continua y acotada y 
\[
w_{k+1}\left(x\right)=\min_{a\in A(x)}\left\{ c\left(x,a\right)+\beta w_{k}\left(f(x,a)\right)\right\} ,
\]

y para cada etapa $k$, se calculan las funciónes $h_{k}$

\[
h_{k}\left(x\right)=\arg\min_{a\in A(x)}\left\{ c\left(x,a\right)+\beta v\left(f(x,a)\right)\right\} ,v=w_{k+1}.
\]

Esto nos da politicas óptimas infinitas 
\[
\pi^{*}(\mathbf{x})=\left\{ h_{k}(x)\right\} _{x\in\mathbf{x}}
\]

donde $x_{0}$ fue dado y $x_{k+1}=f(x_{k},h_{k}(x_{k}))$. Entonces
cortaremos la pólitica cuando $\|w_{k+1}-w_{k}\|<\epsilon,$ donde
$\epsilon$ es un criterio de paro. 

\chapter{Problema 1: Decisiones de Ahorro}

\section{Planteamiento del problema. }

Supongamos que Ricardo ha ganado el premio mayor de la Lotera Nacional
y decide meter su dinero al banco que le ofrece una tasa de interes
anual $i$. Al inicio del año $k$ decide retirar $a$ pesos para
sus gastos y planea repetir este proceso $N$ veces, es decir, $k=0,1,\ldots,N$
1. Si $x_{k}$ denota la cantidad de dinero al inicio del año $k$,
entonces la siguiente ecuacion

\[
x_{k+1}=\left(1+i\right)\left(x_{k}-a_{k}\right)
\]

describe como va cambiando la fortuna de Ricardo en función de los
retiros que haga. Utility theory, una empresa de consultoria, le recomienda
a Ricardo que escoja $a_{0},\ldots,a_{N-1}$ de tal manera que maximice. 

\[
\beta^{N}\left(x_{N}\right)^{1-\gamma}+\sum_{k=0}^{N-1}\beta^{k}\left(a_{k}\right)^{1-\gamma}.
\]

La cantidad final $x_{N}$ será heredada a sus descendientes. Los
parámetros $\beta,\gamma\in\left(0,1\right)$ fueron estimados por
Utility Theory. Usando el APD encontraos que cada función $J_{k}$
es de la forma $J_{k}\left(x\right)=A_{k}\beta^{k}x^{1-\gamma},$
donde $A_{N}=1$ y para $k=N-1,\ldots,0$. 
\[
A_{k}=\left[1+\left[\left(1+i\right)^{1-\gamma}\beta A_{k+1}\right]^{1/\gamma}\right]^{\gamma}
\]


\subsubsection*{Bosquejo de Corrección de $A_{k}$ (Falta la demostración por inducción)}

Considerando $J_{N}$ como sigue

\[
J_{N}\left(x\right)=\beta^{N}x^{1-\gamma}K_{N},
\]

con $K_{N}=1$ bajo la hipótesis de que 
\[
c_{k}\left(x,a\right)=\beta^{k}a^{1-\gamma}
\]
 calculamos $J_{N-1}$. 

\begin{align*}
J_{N-1}\left(x\right) & =\max_{a\in A\left(x\right)}\left\{ c_{N-1}(x,a)+J_{N}\left((1+i)(x-a)\right)\right\} \\
 & =\max_{a\in A\left(x\right)}\left\{ \beta^{N-1}a^{1-\gamma}+\beta^{N}\left((1+i)(x-a)\right)^{1-\gamma}\right\} 
\end{align*}

Definimos el argumento como una función $q$.

\begin{align*}
q(x,a) & =\beta^{N-1}a^{1-\gamma}+\beta^{N}\left((1+i)(x-a)\right)^{1-\gamma}\\
 & =C_{1}a^{1-\gamma}+C_{2}\left(x-a\right)^{1-\gamma},
\end{align*}

donde $C_{1}=\beta^{N-1}$ y $C_{2}=\beta^{N}(1+i)^{1-\gamma}K_{N}.$
Como $q$ es continua en $\left(x,a\right)$. Podemos calcular el
máximo mediante el gradiente. 

\[
\partial_{a}q=C_{1}\left(1-\gamma\right)a^{-\gamma}-C_{2}(1-\gamma)\left(x-a\right)^{-\gamma}.
\]

Igualando, $\partial_{a}q=0$. 
\begin{align*}
C_{1}a^{-\gamma} & =C_{2}\left(x-a\right)^{-\gamma}\\
\dfrac{C_{1}}{C_{2}} & =\left(\dfrac{x-a}{a}\right)^{-\gamma}\\
\left(\dfrac{C_{1}}{C_{2}}\right)^{-\frac{1}{\gamma}} & =\frac{x}{a}-1\\
\left(\dfrac{C_{1}}{C_{2}}\right)^{-\frac{1}{\gamma}}+1 & =\frac{x}{a}\\
a & =\dfrac{x}{\left(\dfrac{C_{1}}{C_{2}}\right)^{-\frac{1}{\gamma}}+1}
\end{align*}

Finalmente 
\[
a=h(x)=\dfrac{x}{\left(\beta(1+i)^{1-\gamma}\right)^{\frac{1}{\gamma}}+1}
\]

Definiendo $\eta=\left(\beta(1+i)^{1-\gamma}\right)^{\frac{1}{\gamma}}+1,$
$\eta-1=\left(\beta(1+i)^{1-\gamma}\right)^{\frac{1}{\gamma}}$

entonces 
\[
h(x)=\dfrac{x}{\eta},
\]

\begin{align*}
J_{N-1}(x) & =\beta^{N-1}\left(\dfrac{x}{\eta}\right)^{1-\gamma}+\beta^{N}\left((1+i)\left(x-\dfrac{x}{\eta}\right)\right)^{1-\gamma}\\
 & =\beta^{N-1}x^{1-\gamma}\left(\eta^{\gamma-1}+\beta\left(1+i\right)^{1-\gamma}\left(\dfrac{\eta-1}{\eta}\right)^{1-\gamma}\right)\\
 & =\beta^{N-1}x^{1-\gamma}\eta^{\gamma-1}\left(1+\beta\left(1+i\right)^{1-\gamma}\left(\eta-1\right)^{1-\gamma}\right)\\
 & =\beta^{N-1}x^{1-\gamma}\eta^{\gamma-1}\left(1+\beta\left(1+i\right)^{1-\gamma}\left(\eta-1\right)^{1-\gamma}\right)\\
 & =\beta^{N-1}x^{1-\gamma}\eta^{\gamma-1}\left(1+\beta\left(1+i\right)^{1-\gamma}\left(\left(\beta(1+i)^{1-\gamma}\right)^{\frac{1}{\gamma}}\right)^{1-\gamma}\right)\\
 & =\beta^{N-1}x^{1-\gamma}\eta^{\gamma-1}\left(1+\beta\left(1+i\right)^{1-\gamma}\left(\beta(1+i)^{1-\gamma}\right)^{\frac{1}{\gamma}-1}\right)\\
 & =\beta^{N-1}x^{1-\gamma}\eta^{\gamma-1}\left(1+\beta^{\frac{1}{\gamma}}(1+i)^{\left(1-\gamma\right)\left(\frac{1}{\gamma}-1\right)+1-\gamma}\right)\\
 & =\beta^{N-1}x^{1-\gamma}\eta^{\gamma-1}\left(1+\beta^{\frac{1}{\gamma}}(1+i)^{\left(\frac{1}{\gamma}-1\right)}\right)\\
 & =\beta^{N-1}x^{1-\gamma}\eta^{\gamma},
\end{align*}

Entonces 
\[
K_{N-1}=\eta^{\gamma},h_{k-1}\left(x\right)=\dfrac{x}{\left(K_{N-1}\right)^{1/\gamma}}
\]

Ahora calculamos $J_{N-2}$

\begin{align*}
J_{N-2}\left(x\right) & =\max_{a\in A\left(x\right)}\left\{ \beta^{N-2}a^{1-\gamma}+\beta^{N-1}\left[\left(1+i\right)\left(x-a\right)\right]^{1-\gamma}\eta^{\gamma}\right\} \\
 & =\max_{a\in A\left(x\right)}\left\{ q\left(x,a\right)\right\} ,
\end{align*}

donde 
\[
q\left(x,a\right)=C_{1}a^{1-\gamma}+C_{2}\left(x-a\right)^{1-\gamma},
\]

con $C_{1}=\beta^{N-2}$ y $C_{2}=\beta^{N-1}\left(1+i\right)^{1-\gamma}K_{N-1}$
. Obteniendo, por recursividad 
\begin{align*}
h_{N-2} & =\dfrac{x}{\left(\dfrac{C_{1}}{C_{2}}\right)^{-\frac{1}{\gamma}}+1}\\
 & =\dfrac{x}{\left(\dfrac{1}{\beta\left(1+i\right)^{1-\gamma}K_{N-1}}\right)^{-\frac{1}{\gamma}}+1}\\
 & =\dfrac{x}{\left(\beta\left(1+i\right)^{1-\gamma}K_{N-1}\right)^{\frac{1}{\gamma}}+1}
\end{align*}

Entonces, sea 
\begin{align*}
\eta' & =\left(\beta\left(1+i\right)^{1-\gamma}K_{N-1}\right)^{\frac{1}{\gamma}}+1.
\end{align*}

Repitiendo, el caso anterior, tenemos que 
\begin{align*}
J_{N-2}\left(x\right) & =\beta^{N-2}x^{1-\gamma}\eta_{'}^{\gamma-1}\left(1+K_{N-1}\beta\left(1+i\right)^{1-\gamma}\left(\left(\beta(1+i)^{1-\gamma}K_{N-1}\right)^{\frac{1}{\gamma}}\right)^{1-\gamma}\right)\\
 & =\beta^{N-2}x^{1-\gamma}\eta_{'}^{\gamma-1}\left(1+K_{N-1}\beta\left(1+i\right)^{1-\gamma}\left(\left(\beta(1+i)^{1-\gamma}K_{N-1}\right)^{\frac{1}{\gamma}}\right)^{1-\gamma}\right)\\
 & =\beta^{N-2}x^{1-\gamma}\eta_{'}^{\gamma-1}\left(1+K_{N-1}\beta\left(1+i\right)^{1-\gamma}\left(\beta(1+i)^{1-\gamma}K_{N-1}\right)^{\frac{1}{\gamma}-1}\right)\\
 & =\beta^{N-2}x^{1-\gamma}\eta_{'}^{\gamma-1}\left(1+K_{N-1}\beta\left(1+i\right)^{1-\gamma}(1+i)^{\left(1-\gamma\right)\left(\frac{1}{\gamma}-1\right)}K_{N-1}^{\frac{1}{\gamma}-1}\right)\\
 & =\beta^{N-2}x^{1-\gamma}\eta_{'}^{\gamma-1}\left(1+K_{N-1}\beta^{1/\gamma}\left(1+i\right)^{\frac{1}{\gamma}-1}K_{N-1}^{\frac{1}{\gamma}-1}\right)\\
 & =\beta^{N-2}x^{1-\gamma}\eta_{'}^{\gamma-1}\left(1+\beta^{1/\gamma}\left(1+i\right)^{\frac{1}{\gamma}-1}K_{N-1}^{\frac{1}{\gamma}}\right)\\
 & =\beta^{N-2}x^{1-\gamma}\eta'{}^{\gamma},
\end{align*}

entonces 
\[
K_{N-2}=\eta'{}^{\gamma},
\]

y 
\[
h_{N-2}=\dfrac{x}{K_{N-2}^{1/\gamma}}
\]

Por lo tanto, tenemos que 
\[
K_{n}=\left(\beta\left(1+i\right)^{1-\gamma}K_{n+1}\right)^{\frac{1}{\gamma}}+1,n=0,1,2,\ldots,N,
\]
con $K_{N}=1$. Obteniendo así 
\begin{align*}
J_{n}\left(x\right) & =\beta^{n}x^{1-\gamma}K_{n}\\
h_{n}\left(x\right) & =\dfrac{x}{K_{n}^{1/\gamma}}
\end{align*}


\section{Planteamiento de la Implementación. }

Notemos que en nuestro caso $\mathcal{X},\mathcal{A}$ son espacios
finitos. ya que para $x_{0}\in\mathcal{X}$ dado solo tomaremos $N$
acciones. En el problema del ahorro $A\left(x\right)=[0,x]$, es decir,
solo podemos retirar a lo más lo que tenemos en ese momento. Luego,
identificamos que la función recompensa (Nombre habitual cuando queremos
maximizarla, función costo en el caso contrario)
\[
R\left(x_{0},\pi\right)=\beta^{N}\left(x_{N}\right)^{1-\gamma}+\sum_{k=0}^{N-1}\beta^{k}\left(a_{k}\right)^{1-\gamma},
\]

donde $\pi=\left(a_{0},a_{1},\ldots,a_{N-1}\right)$. Por definición, 

\chapter{El problma del lago <de Dechert y O'Donnell.}

En terminos economicos, los lagos son de vital importancia en una
comunidad agrcola. En parte, proveen de agua, peces, zonas recreativas
y aumentan la plusvala de complejos residenciales. Por otro lado,
son parte medular de la actividad agrcola pues sirven como vertedero
de desechos. 

Los desechos agrcolas son ricos en fosforo: nutriente principal para
hierbas y algas marinas. En consecuencia, arrojar cantidades signicativas
de fosforo crea condiciones favorables para el crecimiento y reproduccion
de algas. Las algas consumen oxgeno y secretan toxinas, por ello,
su sobrepoblacion afecta el desarrollo de peces, haciendo el lago
inseguro para usarlo como zona recreativa y al mismo tiempo, deteriora
la plusvala de zonas residenciales. En resumen, la descarga de fosforo
en un lago, afecta su benecio economico. 

En contraste, de forma indirecta, dichas descargas forman parte de
la actividad agropecuaria y por consiguiente tambien se relacionan
con la utilidad economica. Dechert y O'Donnell plantean un problema
de optimizacion estocastica, para calcular la cantidad de fosforo
optima a descargar en un lago, que maximice la utilidad generada por
la actividad agrcola. En esta seccion se considera la version determinista.
Sea $x_{k}$ el nivel de fosforo en la etapa k del lago y a la correspondiente
descarga de fosforo. Empleando una funcion de utilidad usada por ecologistas,
Dechert y O'Donell proponen el siguiente problema de optimización
descontado 

\[
\max\sum_{k=0}^{\infty}\beta^{k}\left(\log\left(a_{k}+\text{eps}\right)-\kappa x_{k}^{2}\right),
\]
sujeto a 
\[
x_{k+1}=bx_{k}+\dfrac{x_{k}^{q}}{1+x_{k}^{q}}+a,
\]

donde el parámetro $b$ representa la fracción de fósforo que queda
en el lago de un tiempo $k$ hasta un tiempo $k+1$, el parámetro
$q$ está asociado a un punto.

\section{Planteamiento Pre-Implementación}

En este tipo de problemas, lo primero es definir la función de costo,
en el caso del lago tenemos que 
\end{document}
